% Author: Juneleung
% Date  : 2023-08-01
% Contact : Juneleungchan@163.com
% Github: https://github.com/juneleung/BFA_Master_Thesis_Template

\usepackage{xeCJK}
\documentclass[UTF8,a4paper,12pt]{ctexart}
\input{formatSetting.tex}

% ----------------------------------------

% 学科(专业领域)和研究方向的名称参照培养计划填写。
\newcommand{\专业领域}{输入专业领域名称} % 如“电影学专业”、“电影领域”
\newcommand{\研究方向}{输入研究方向名称} % 如“电影表演创作及理论方向”、“电影声音创作方向”
\newcommand{\论文类型}{全日制艺术硕士学位论文}
\newcommand{\年级}{(2022届)}

% ----------------------------------------

% 3.标题
% 要求确切、具体。即题文一致、题文相符,切忌过于笼统、空泛。标题应概括论文的核心范畴和论点,一般不超过20个字。如标题过长,可用副标题补充。

\newcommand{\论文中文标题}{输入论文中文标题}
\newcommand{\论文中文副标题}{——输入论文中文副标题(若无请留空"\ ")} 
% 如:
% \newcommand{\论文中文副标题}{\ }
\newcommand{\论文英文标题}{ENGLISH TITLE (16pt,Time New Roman,Bold)}
\newcommand{\论文英文副标题}{——输入论文英文副标题(若无请留空"\ ")}

% ----------------------------------------

\newcommand{\研究生姓名}{输入作者姓名}
\newcommand{\导师姓名I}{输入导师姓名}
\newcommand{\导师姓名II}{导师 2(若无请留空"\ ")}
\newcommand{\导师姓名III}{导师 3(若无请留空"\ ")}
\newcommand{\答辩时间年}{2023}
\newcommand{\答辩时间月}{05}
\newcommand{\答辩时间日}{01}
 
% ----------------------------------------
% ----------------------------------------
% ----------------------------------------

% \linenumbers % 行号


\newcommand{\makeCoverPage}{



% ----------------------------------------

% 1.封面
% 采用研究生院提供的统一格式。包括:学科(专业领域)名称,如“电影学专业”、“电影领域”;研究方向名称,如“电影表演创作及理论方向”、“电影声音创作方向”;学位论文题目;作者姓名;导师姓名;学位论文完成时间等。
% 学科(专业领域)和研究方向的名称参照培养计划填写。

% ----------------------------------------



% 封面页
\begin{center}\hei

    \vspace*{10.5pt}

    \begin{figure}[htbp]
        \centering
        \includegraphics[width=10.08cm,height=1.68cm]{pic/bfalogo.png}
        \label{fig:bfalogo}
    \end{figure}

    \vspace*{10.5pt}
    \vspace*{10.5pt}
    \vspace*{10.5pt}

    {\zihao{-2}\bfseries  \专业领域\\}
    {\zihao{-2}\bfseries  \研究方向\\}
    {\zihao{-2}\bfseries  \论文类型\\}
    {\zihao{-2}\bfseries  \年级\\}
    
    \vspace*{10.5pt}
    \vspace*{10.5pt}

    {\zihao{2}\bfseries  \论文中文标题\\}
    {\zihao{3}\bfseries  \论文中文副标题\\}
    
    \vspace*{10.5pt}
    \vspace*{10.5pt}
    
    {\zihao{3}\bfseries \TimesB \论文英文标题\\}
    {\zihao{3}\bfseries  \论文英文副标题\\}
    
    \vspace*{10.5pt}
    \vspace*{10.5pt}
    \vspace*{10.5pt}
    \vspace*{10.5pt}
    
    \begin{tabular}{>{\hei\bfseries\fontsize{16pt}{16pt}}r>{\fontsize{16pt}{16pt}\bfseries}l}
        研究生:    &   \研究生姓名\\[10pt]
        导 师:    &   \导师姓名I\\[10pt]
                &   \导师姓名II\\[10pt]
                &   \导师姓名III\\[10pt]
        ~\\[10pt]
        ~\\[10pt]
        答辩委员会主席:    &  \underline{   }\\[10pt]
        答辩委员会成员:    &  \underline{   } \underline{   } \underline{   } \underline{   }\\[10pt]
           答 辩 时 间:    &  \underline{\答辩时间年} 年 \underline{\答辩时间月} 月 \underline{\答辩时间日} 日 \\[10pt]
    \end{tabular}

\end{center}
\thispagestyle{empty}
}                             % 封面
\usepackage{wasysym}


% ----------------------------------------

% 2.声明
% 采用研究生院提供的统一格式。声明位于学位论文首页,包括《独创性声明》和《学位论文使用授权书》,须由研究生和导师本人用黑色中性笔亲笔签名、填写日期。

% ----------------------------------------




\newcommand{\makeStatement}{
\newpage
% 独创性声明
\begin{spacing}{1.3}\songti\zihao{-4}
    \begin{center}
        \bfseries 独创性声明
    \end{center}
        \par 本人郑重声明:所呈交的学位论文,是本人在导师的指导下独立进行研究工作所取得的成果。除文中已经注明引用的内容外,本论文不含任何其他个人或集体已经发表或撰写过的作品成果。对本文的研究做出重要贡献的个人和集体,均已在文中以明确方式标明。本人完全意识到本声明的法律结果由本人承担。\\
        ~\\
        \par 学位论文作者签名:{\underline{\qquad\qquad\qquad}}
        \qquad
        日期:{\underline{\qquad\qquad\qquad}}\\
        ~\\
    
    \begin{center}
        \bfseries 学位论文使用授权书
    \end{center}

    \par 学位论文作者完全了解北京电影学院关于保存、使用学位论文的规定,即学校有权保留并向国家有关部门或机构送交学位论文的复印件和电子版,允许学位论文被查阅和借阅;学校可以公布学位论文的全部或部分内容,可以采用影印、缩印或其他复制手段保存、汇编学位论文。保密的学位论文在解密后遵守此规定。

    保密论文注释:本学位论文属于保密在\underline{\qquad}年解密后适用本授权书。

    非保密论文注释:本学位论文不属于保密范围,适用本授权书。

    学位论文全文电子版同意提交后:\Square 一年 \Square 二年在校园网上发布,供校内师生浏览。
    ~\\

    \par 本人签名:{\underline{\qquad\qquad\qquad}}
    \qquad
    日期:{\underline{\qquad\qquad\qquad}}\\
    \par 导师签名:{\underline{\qquad\qquad\qquad}}
    \qquad
    日期:{\underline{\qquad\qquad\qquad}}

\end{spacing}
\thispagestyle{empty}
}                         % 声明

\begin{document}

\makeCoverPage % 封面
\makeStatement % 声明

\newpage


% ----------------------------------------

% 4.中英文摘要和关键词
% 中文摘要及英文摘要各一份,中文摘要在前,英文摘要在后。
% 摘要是学位论文的内容不加注释和评论的简短陈述。摘要应具有独立性和自含性,一般应说明研究目的、研究方法、结果和最终结论等,重点是结果和结论。注意突出具有创新性的成果和新见解。
% 关键词是为了文献标引而从学术论文中选取出来用以表示全文主题内容信息款目的单词或术语。
% (1)硕士学位论文的中文摘要1000—1500字左右。博士学位论文的中文摘要1500—2000字左右。
% 论文题目为黑体三号,可以分成1—2行居中打印。
% 论文题目下空一行,输入“摘要”二字,黑体小三,两字中间空一格(“一格”的标准为一个汉字,以下同),居中。
% “摘要”二字下空一行,输入中文摘要内容(宋体小四)。段落按照“首行缩进”格式,每段开头空二格,标点符号占一格。
% 摘要内容后下空一行,输入“关键词:”(黑体四号);其后输入3—5个关键词(宋体小四),各词之间空一格。
% (2)学位论文英文摘要,一般为A4纸1—2页:
% 学位论文(包括摘要和正文)中的英文一律采用“Times New Roman”字体。
% 论文英文题目全部采用大写字母,可分成1—3行居中打印。每行左右两边至少留五个字符空格。
% 英文题目下空三行,输入“ABSTRACT”,小三号“Times New Roman”字体,居中。
% 再下空一行,输入英文摘要内容(小四号“Times New Roman”字体)。英文摘要与中文摘要相对应。
% 英文摘要内容下空二行,输入“KEY WORDS:”(四号“Times New Roman”字体),其后输入3—5个关键词(四号“Times New Roman”字体),英文小写,中间用空格隔开。

% ----------------------------------------

\setcounter{secnumdepth}{0}
\setcounter{page}{1}
\pagenumbering{Roman}

% 中文摘要
\hei\zihao{3}\noindent\begin{center}
    \论文中文标题\\
    \论文中文副标题
\end{center}

\section{\hei\zihao{-3}摘\quad 要}
{
    \fsong\zihao{-4}
        输入摘要内容
        
        (摘要是学位论文的内容不加注释和评论的简短陈述。摘要应具有独立性和自含性,一般应说明研究目的、研究方法、结果和最终结论等,重点是结果和结论。注意突出具有创新性的成果和新见解。硕士要求1000-1500字)

        % ~\\~\\~\\~\\~\\~\\~\\~\\~\\
        ~\\
    
    \hei\zihao{4}\noindent
        关键词:
    \fsong\zihao{-4}
        关键词1\quad 关键词2\quad 关键词3

}
\thispagestyle{empty}
\newpage

% 英文摘要
\hei\zihao{3}\noindent\begin{center}
    \论文英文标题\\
    \论文英文副标题
\end{center}

~\\~\\

\section{\zihao{-3} ABSTRACT}
{
    ~\\
    
    \zihao{-4}
    Enter abstract content...
            
    Enter abstract content...

    Enter abstract content...
        

    ~\\~\\
    \zihao{4}\noindent
    KEY WORDS: word1 word2 word3
        
}
\thispagestyle{empty}                          % 摘要
\newpage

% ----------------------------------------

% 5.目录
% “目录”二字为黑体三号,两字中间空一格,居中。
% 再下空两行。目录页每行均由标题名称和页码组成,排列顺序是:绪论(或前言);正文序号和标题;参考文献;附录;致谢等。目录页的正文内容排版到三级标题(章、条和款),须与正文保持一致。
% 注意:学位论文完成后,在WORD中执行“插入/引用/索引和目录”命令自动生成目录。

% ----------------------------------------


% 目录
\addcontentsline{toc}{section}{\hei\zihao{3}目\quad 录}
\thispagestyle{empty}
% \section[目录]{}
\tableofcontents
\thispagestyle{empty}                   % 目录
\newpage

% ----------------------------------------

% 6.正文
% 正文是学位论文的核心部分,一般包括绪论(或前言)、本论和结语。博士学位论文的总字数应不少于8万字,原则上不多于15万字;学术学位硕士和同等学力人员的学位论文应不少于3万字,原则上不多于5万字;全日制艺术硕士(MFA)和在职艺术硕士(MFA)的学位论文应不少于1万字,是对学位作品的创作思考、理论阐释或理论批评。
% (1)绪论(或前言)
% 简要说明本研究工作的目的和范围;相关研究领域中的国内外研究现状以及前人所做的工作;论文的主题和理论基础;论文的研究方向、研究路线和方法;论文的理论和实践意义等。“绪论”(或“前言”)二字为黑体三号,两字中间空一格,居中。绪论(或前言)内容为宋体小四。

% ----------------------------------------



% 引言
\setcounter{secnumdepth}{0}
\setcounter{page}{1}
\pagenumbering{arabic}

% \setcounter{section}{0}
\section{\hei\zihao{3}绪\quad 论}

    \song\zihao{-4}
    输入绪论正文内容。

    输入绪论正文内容。



\setcounter{secnumdepth}{4}
                      % 引言/绪论




\newpage
% ----------------------------------------

% (2)本论
% 本论是包括研究中的理论分析、图片资料、调查对象、主要论点、论据和结论,总体要求文字通畅、论证有力、逻辑严谨、引用规范。以下两种格式任选其一。
% A.章、条、款、项层次代号及格式一:
% 1(章的标题)XXXX,黑体三号,居中,段前、段后各空2行。对应样式:标题1;
% 输入第1章正文内容,宋体小四。
% 1.1(条的标题)XXXX,黑体小三,居中,段前、段后各空1行。对应样式:标题2;
% 输入第1章第1条正文内容,宋体小四。
% 1.1.1(款的标题)XXXX,黑体四号,居左,段前、段后各空0.5行。对应样式:标题3;
% 输入第1章第1条第1款正文内容,宋体小四。
% 1.1.1.1(项的标题)XXXX,黑体小四,居左,段前、段后各空0.5行。对应样式:标题4;
% 输入第1章第1条第1款第1项正文内容,宋体小四。
% ……
% B.章、条、款、项层次代号及格式二:
% 第一章(章的标题)XXXX,黑体三号,居中,段前、段后各空2行。对应样式:标题1;
% 输入第1章正文内容,宋体小四。
% 第一节(条的标题)XXXX,黑体小三,居左,段前、段后各空1行。对应样式:标题2;
% 输入第1章第1条正文内容,宋体小四。
% 一、(款的标题)XXXX,黑体四号,居左,段前、段后各空0.5行。对应样式:标题3;
% 输入第1章第1条第1款正文内容,宋体小四。
% 1.(项的标题)XXXX,黑体小四,居左,段前、段后各空0.5行。对应样式:标题4;
% 输入第1章第1条第1款第1项正文内容,宋体小四。
% ……

% ----------------------------------------


\section{输入第一章的标题} % (fold)

    ~\\

    \song\zihao{-4}
    输入第一章正文内容,宋体小四。  


    \subsection{输入此条标题} % (fold)

        输入第一章第一条正文内容,宋体小四。

        \subsubsection{输入此款标题} % (fold)

            输入第1章第1条第1款正文内容,宋体小四。

                \subsubsubsection{输入此项标题} % (fold)

                输入第1章第1条第1款正文内容,宋体小四。



                
\subsection{论文工作内容} % (fold)

    ~\\
    
    论文工作内容

    \subsubsection{输入此款标题} % (fold)

        输入第1章第1条第1款正文内容,宋体小四。

            \subsubsubsection{输入此项标题} % (fold)

            输入第1章第1条第1款正文内容,宋体小四。

        正文内容

        \begin{figure}[ht]
            \centering
            \includegraphics[width=0.8\textwidth]{example-image} 
            \caption{图片标题\protect} % \footnotemark
            \label{img} 
        \end{figure}
        
        % ----------------------------------------
        
        %  12.图、表和公式
        % (1)图:图的编号由“图”和从1开始的阿拉伯数字组成,如“图1”、“图2”等。图的编号应一直连续到附录之前,并与正文的章、条和表的编号无关。只有一幅图时,仍应标为“图1”。图应有图题,置于图的编号之后空一格。图的编号和图题应置于图下方的居中位置。
        % (2)表:表的编号由“表”和从1开始的阿拉伯数字组成,如“表1”、“表2”等。表的编号应一直连续到附录之前,并与正文的章、条和图的编号无关。只有一张表时,仍应标为“表1”。表应有表题,置于表的编号之后空一格。表的编号和表题应置于表上方的居中位置。
        % (3)公式:公式序号一律采用阿拉伯数字分章依序编排;如:式(2-13)、式(4-5),其标注应于该公式所在行的最右侧;公式书写方式应在文中相应位置另起一行居中横排,对于较长的公式只可在符号处(+、-、*、/、≤、≥等)转行。
            
        % ----------------------------------------



        正文内容,如图\ref{img}所示。
    
        引用参考\upcite{ref_cGAN_2014}文献。

        引用参考文献电影工业美学\cite{电影工业美学}。
        






                      % 第一章

\newpage
\section{输入第二章的标题} % (fold)

~\\
    
    \song\zihao{-4}
    输入第二章正文内容,宋体小四。  


    \subsection{输入此条标题} % (fold)

        输入第二章第一条正文内容,宋体小四。

        \subsubsection{输入此款标题} % (fold)

            输入第2章第1条第1款正文内容,宋体小四。

                \subsubsubsection{输入此项标题} % (fold)

                输入第1章第1条第1款正文内容,宋体小四。




\subsection{第二章第二小节} % (fold)

~\\
    
    第二章第二小节

    \subsubsection{输入此款标题} % (fold)

        输入第2章第2条第1款正文内容,宋体小四。

            \subsubsubsection{输入此项标题} % (fold)

            输入第1章第1条第1款正文内容,宋体小四。







                      % 第二章

\newpage
\section{输入第三章的标题} % (fold)

~\\
    
    \song\zihao{-4}
    输入第三章正文内容,宋体小四。  


    \subsection{输入此条标题} % (fold)

        输入第三章第一条正文内容,宋体小四。

        \subsubsection{输入此款标题} % (fold)

            输入第3章第1条第1款正文内容,宋体小四。

                \subsubsubsection{输入此项标题} % (fold)

                输入第1章第1条第1款正文内容,宋体小四。




\subsection{第三章第二小节} % (fold)

~\\
    
    第三章第二小节

    \subsubsection{输入此款标题} % (fold)

        输入第3章第2条第1款正文内容,宋体小四。

            \subsubsubsection{输入此项标题} % (fold)

            输入第1章第1条第1款正文内容,宋体小四。







                      % 第三章


\newpage


% ----------------------------------------

% (3)结语
% 对学位论文中提出问题、解决问题和所得出结论的总结性论述。“结语”二字为黑体三号,两字中间空一格,居中。结语内容为宋体小四。


% ----------------------------------------




% 结语
% \setcounter{section}{0}
\section*{\hei\zihao{3}结\quad 语}

    \song\zihao{-4}
    输入结语内容。

    输入结语内容。


\setcounter{secnumdepth}{3}

                   % 结语


\newpage

% ----------------------------------------

% 7.参考文献
% 参考文献是对论文引文进行统计和分析的重要信息源之一。
% 参考文献著录项目包括:主要责任者;文献题名及版本(初版省略);文献类型及载体类型标识;出版项(出版地、出版者、出版年);文献出处或电子文献的可获得地址;文献起止页码;文献标准编号(标准号、专利号等)。
% 表1 参考文献类型及载体类型标识
% 参考文献类型	文献类型标识	参考文献类型	文献类型标识
% 专  著	M	电子公告	EB
% 论文集	C	磁  带	MT
% 报纸文章	N	磁  盘	DK
% 期刊文章	J	光  盘	CD
% 学位论文	D	联机网络	OL
% 报  告	R	联机网上数据库	DB/OL
% 标  准	S	磁带数据库	DB/MT
% 专  利	P	光盘图书	M/CD
% 专著、论文集中析出的文献	A	磁盘软件	CP/DK
% 其他未说明的文献类型	Z	网上期刊	J/OL
% 数据库	DB	网上电子公告	EB/OL
% 计算机程序	CP		
% (1)学位论文中引用参考文献
% 在引出处的右上方用阿拉伯数字编排的序号;脚注一律用阿拉伯数字序号(即①②③…)标注,居于页面底端,每页重新编号,中文使用宋体小五号,英文采用Times New Roman。文中以脚注出现的引注参考文献需要标示文献起止页码。
% (2)论文末附的参考文献
% “参考文献”四字用黑体三号,居中。参考文献内容用宋体小四。编排格式取左对齐,以阿拉伯数字连续编号(即[1][2][3]…)标注,按照出版年代排列(由远及近或由近及远)。论文末附的参考文献不用标示文献起止页码。
% (3)参考文献条目的编排格式
% 各类参考文献条目的编排格式如下:
% a.专著、论文集、学位论文、报告
% 序号.主要责任者.文献题名[文献类型标识].出版地:出版者,出版年:起止页码.
% b.期刊文章
% 序号.主要责任者.文献题名[J].刊名,年,卷(期):起止页码.
% c.论文集中的析出文献
% 序号.析出文献主要责任者.析出文献题名[A].原文献主要责任者(任选).原文献题名[C].出版地:出版者,出版年:起止页码.
% d.报纸文章
% 序号.主要责任者.文献题名[N].报纸名,出版日期(版次).
% e.国际、国家标准
% 序号.标准编号,标准名称[S].
% f.专利
% 序号.专利所有者.专利题名[P].专利国别:专利号,出版日期.
% g.电子文献
% 序号.主要责任者.电子文献题名[电子文献及载体类型标识].电子文献的出处或可获得地址,发表或更新日期/引用日期(任选).
% h.各种未定义类型的文献
% 序号.主要责任者.文献题名[Z].出版地:出版者,出版年.

% ----------------------------------------



\setcounter{secnumdepth}{0}
% \section[参考文献]{}

\zihao{5}{
\begin{thebibliography}{99}
\addcontentsline{toc}{section}{参考文献}

% 1 GAN
% \bibitem{ref_GAN_2014}
% Ian J. Goodfellow, Jean Pouget-Abadie, Mehdi Mirza, Bing Xu, David Warde-Farley, Sherjil Ozair, Aaron Courville, Yoshua Bengio.
% Generative Adversarial Nets[J]. 
% \emph{In Proceedings of the 27th International Conference on Neural Information Processing Systems - Volume 2 (NIPS)}, 2014, 2672–2680.

% 2 CGAN
\bibitem{ref_cGAN_2014}
Mehdi Mirza, Simon Osindero.
Conditional Generative Adversarial Nets[J].
\emph{CoRR Abs}, 2014.

\bibitem{电影工业美学}
陈旭光.论电影工业美学的学科体系建构与方法论意识[J/OL].东岳论丛,2023(07):24-31+191[2023-08-01].DOI:10.15981/j.cnki.dongyueluncong.2023.07.003.

% [1] 序号.主要责任者.文献题名[文献类型标识].出版地:出版者,出版年:起止页码.
% [2] 序号.主要责任者.文献题名[J].刊名,年,卷(期):起止页码.
% [3] 剩下的格式参考一下那个说明文档吧


\end{thebibliography}
}
                   % 参考文献


% \bibliographystyle{acm}
% \bibliography{citation}

\newpage

% ----------------------------------------

% 8.附录
% 附录一般作为学位论文主体的补充项目。主要包括正文内过于冗长的公式推导;供读者阅读方便所需要的辅助性的数学工具或重复性数据图表;由于过分冗长而不宜放置在正文中的计算机程序清单;本专业内具有参考价值的资料;论文使用的缩写说明等。附录编于正文后,其页码与正文连续编排。
% “附录”二字为黑体三号,两字中间空一格,居中。附录内容用宋体小四。
% 论文的附录依序编排为附录1,附录2…。附录中的图表公式另编排序号,与正文分开。

% ----------------------------------------



% 附录
% \setcounter{section}{0}
\section*{\hei\zihao{3}附\quad 录}

    \song\zihao{-4}
    输入附录正文内容
    
    (附录一般作为学位论文主体的补充项目。主要包括正文内过于冗长的公式推导;供读者阅读方便所需要的辅助性的数学工具或重复性数据图表;由于过分冗长而不宜放置在正文中的计算机程序清单;本专业内具有参考价值的资料;论文使用的缩写说明等。附录编于正文后,其页码与正文连续编排。)




\setcounter{secnumdepth}{4}

                          % 附录
\newpage

% ----------------------------------------

% 9.致谢
% 对于提供各类资助、指导和协助完成论文研究工作的单位及个人表示感谢。致谢应实事求是,真诚客观。“致谢”二字用黑体三号,两字中间空一格,居中。致谢内容用宋体小四。

% 10.注释
% 注释作为脚注在页下分散著录,按照本学科国内外通行的范式,逐一注明本文引用或参考、借用的资料数据出处及他人的研究成果和观点,严禁抄袭剽窃。

% ----------------------------------------



% 致谢
% \setcounter{section}{0}
\section*{\hei\zihao{3}致\quad 谢}

    \song\zihao{-4}
    输入致谢内容
    
    (对于提供各类资助、指导和协助完成论文研究工作的单位及个人表示感谢。致谢应实事求是,真诚客观)。
 



\setcounter{secnumdepth}{4}


                          % 致谢
\newpage

% ----------------------------------------

% 11.作者攻读学位期间发表的学术论文目录
% 按照学术论文发表的时间顺序,列出作者在攻读学位期间发表的或已录用的学术论文清单:包括期刊名称、卷册号、页码、年月及论文署名的排名,并对发表期刊类别、录用以及检索情况做出具体说明。

% ----------------------------------------



% 致谢
% \setcounter{section}{0}
\section*{作者攻读学位期间发表的学术论文目录}

    \song\zihao{-4}
    按照学术论文发表的时间顺序,列出作者在攻读学位期间发表的或已录用的学术论文清单:包括期刊名称、卷册号、页码、年月及论文署名的排名,并对发表期刊类别、录用以及检索情况做出具体说明。




\setcounter{secnumdepth}{4}


                 % 作者攻读学位期间发表的学术论文目录


\end{document}
