
\newpage
% ----------------------------------------

% (2)本论
% 本论是包括研究中的理论分析、图片资料、调查对象、主要论点、论据和结论,总体要求文字通畅、论证有力、逻辑严谨、引用规范。以下两种格式任选其一。
% A.章、条、款、项层次代号及格式一:
% 1(章的标题)XXXX,黑体三号,居中,段前、段后各空2行。对应样式:标题1;
% 输入第1章正文内容,宋体小四。
% 1.1(条的标题)XXXX,黑体小三,居中,段前、段后各空1行。对应样式:标题2;
% 输入第1章第1条正文内容,宋体小四。
% 1.1.1(款的标题)XXXX,黑体四号,居左,段前、段后各空0.5行。对应样式:标题3;
% 输入第1章第1条第1款正文内容,宋体小四。
% 1.1.1.1(项的标题)XXXX,黑体小四,居左,段前、段后各空0.5行。对应样式:标题4;
% 输入第1章第1条第1款第1项正文内容,宋体小四。
% ……
% B.章、条、款、项层次代号及格式二:
% 第一章(章的标题)XXXX,黑体三号,居中,段前、段后各空2行。对应样式:标题1;
% 输入第1章正文内容,宋体小四。
% 第一节(条的标题)XXXX,黑体小三,居左,段前、段后各空1行。对应样式:标题2;
% 输入第1章第1条正文内容,宋体小四。
% 一、(款的标题)XXXX,黑体四号,居左,段前、段后各空0.5行。对应样式:标题3;
% 输入第1章第1条第1款正文内容,宋体小四。
% 1.(项的标题)XXXX,黑体小四,居左,段前、段后各空0.5行。对应样式:标题4;
% 输入第1章第1条第1款第1项正文内容,宋体小四。
% ……

% ----------------------------------------


\section{输入第一章的标题} % (fold)

    ~\\

    \song\zihao{-4}
    输入第一章正文内容,宋体小四。  


    \subsection{输入此条标题} % (fold)

        输入第一章第一条正文内容,宋体小四。

        \subsubsection{输入此款标题} % (fold)

            输入第1章第1条第1款正文内容,宋体小四。

                \subsubsubsection{输入此项标题} % (fold)

                输入第1章第1条第1款正文内容,宋体小四。



                
\subsection{论文工作内容} % (fold)

    ~\\
    
    论文工作内容

    \subsubsection{输入此款标题} % (fold)

        输入第1章第1条第1款正文内容,宋体小四。

            \subsubsubsection{输入此项标题} % (fold)

            输入第1章第1条第1款正文内容,宋体小四。

        正文内容

        \begin{figure}[ht]
            \centering
            \includegraphics[width=0.8\textwidth]{example-image} 
            \caption{图片标题\protect} % \footnotemark
            \label{img} 
        \end{figure}
        
        % ----------------------------------------
        
        %  12.图、表和公式
        % (1)图:图的编号由“图”和从1开始的阿拉伯数字组成,如“图1”、“图2”等。图的编号应一直连续到附录之前,并与正文的章、条和表的编号无关。只有一幅图时,仍应标为“图1”。图应有图题,置于图的编号之后空一格。图的编号和图题应置于图下方的居中位置。
        % (2)表:表的编号由“表”和从1开始的阿拉伯数字组成,如“表1”、“表2”等。表的编号应一直连续到附录之前,并与正文的章、条和图的编号无关。只有一张表时,仍应标为“表1”。表应有表题,置于表的编号之后空一格。表的编号和表题应置于表上方的居中位置。
        % (3)公式:公式序号一律采用阿拉伯数字分章依序编排;如:式(2-13)、式(4-5),其标注应于该公式所在行的最右侧;公式书写方式应在文中相应位置另起一行居中横排,对于较长的公式只可在符号处(+、-、*、/、≤、≥等)转行。
            
        % ----------------------------------------



        正文内容,如图\ref{img}所示。
    
        引用参考\upcite{ref_cGAN_2014}文献。

        引用参考文献电影工业美学\cite{电影工业美学}。
        






