\newpage

% ----------------------------------------

% 8.附录
% 附录一般作为学位论文主体的补充项目。主要包括正文内过于冗长的公式推导;供读者阅读方便所需要的辅助性的数学工具或重复性数据图表;由于过分冗长而不宜放置在正文中的计算机程序清单;本专业内具有参考价值的资料;论文使用的缩写说明等。附录编于正文后,其页码与正文连续编排。
% “附录”二字为黑体三号,两字中间空一格,居中。附录内容用宋体小四。
% 论文的附录依序编排为附录1,附录2…。附录中的图表公式另编排序号,与正文分开。

% ----------------------------------------



% 附录
% \setcounter{section}{0}
\section*{\hei\zihao{3}附\quad 录}

    \song\zihao{-4}
    输入附录正文内容
    
    (附录一般作为学位论文主体的补充项目。主要包括正文内过于冗长的公式推导;供读者阅读方便所需要的辅助性的数学工具或重复性数据图表;由于过分冗长而不宜放置在正文中的计算机程序清单;本专业内具有参考价值的资料;论文使用的缩写说明等。附录编于正文后,其页码与正文连续编排。)




\setcounter{secnumdepth}{4}

