\newpage


% ----------------------------------------

% 4.中英文摘要和关键词
% 中文摘要及英文摘要各一份,中文摘要在前,英文摘要在后。
% 摘要是学位论文的内容不加注释和评论的简短陈述。摘要应具有独立性和自含性,一般应说明研究目的、研究方法、结果和最终结论等,重点是结果和结论。注意突出具有创新性的成果和新见解。
% 关键词是为了文献标引而从学术论文中选取出来用以表示全文主题内容信息款目的单词或术语。
% (1)硕士学位论文的中文摘要1000—1500字左右。博士学位论文的中文摘要1500—2000字左右。
% 论文题目为黑体三号,可以分成1—2行居中打印。
% 论文题目下空一行,输入“摘要”二字,黑体小三,两字中间空一格(“一格”的标准为一个汉字,以下同),居中。
% “摘要”二字下空一行,输入中文摘要内容(宋体小四)。段落按照“首行缩进”格式,每段开头空二格,标点符号占一格。
% 摘要内容后下空一行,输入“关键词:”(黑体四号);其后输入3—5个关键词(宋体小四),各词之间空一格。
% (2)学位论文英文摘要,一般为A4纸1—2页:
% 学位论文(包括摘要和正文)中的英文一律采用“Times New Roman”字体。
% 论文英文题目全部采用大写字母,可分成1—3行居中打印。每行左右两边至少留五个字符空格。
% 英文题目下空三行,输入“ABSTRACT”,小三号“Times New Roman”字体,居中。
% 再下空一行,输入英文摘要内容(小四号“Times New Roman”字体)。英文摘要与中文摘要相对应。
% 英文摘要内容下空二行,输入“KEY WORDS:”(四号“Times New Roman”字体),其后输入3—5个关键词(四号“Times New Roman”字体),英文小写,中间用空格隔开。

% ----------------------------------------

\setcounter{secnumdepth}{0}
\setcounter{page}{1}
\pagenumbering{Roman}

% 中文摘要
\hei\zihao{3}\noindent\begin{center}
    \论文中文标题\\
    \论文中文副标题
\end{center}

\section{\hei\zihao{-3}摘\quad 要}
{
    \fsong\zihao{-4}
        输入摘要内容
        
        (摘要是学位论文的内容不加注释和评论的简短陈述。摘要应具有独立性和自含性,一般应说明研究目的、研究方法、结果和最终结论等,重点是结果和结论。注意突出具有创新性的成果和新见解。硕士要求1000-1500字)

        % ~\\~\\~\\~\\~\\~\\~\\~\\~\\
        ~\\
    
    \hei\zihao{4}\noindent
        关键词:
    \fsong\zihao{-4}
        关键词1\quad 关键词2\quad 关键词3

}
\thispagestyle{empty}
\newpage

% 英文摘要
\hei\zihao{3}\noindent\begin{center}
    \论文英文标题\\
    \论文英文副标题
\end{center}

~\\~\\

\section{\zihao{-3} ABSTRACT}
{
    ~\\
    
    \zihao{-4}
    Enter abstract content...
            
    Enter abstract content...

    Enter abstract content...
        

    ~\\~\\
    \zihao{4}\noindent
    KEY WORDS: word1 word2 word3
        
}
\thispagestyle{empty}